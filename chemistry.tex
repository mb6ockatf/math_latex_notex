\documentclass[a4paper]{article}
\usepackage[utf8]{inputenc}
\usepackage[russian]{babel}
\usepackage{mathtools}
\usepackage{gensymb}
\usepackage{geometry}
\geometry{a4paper,left=6mm,right=6mm,bottom=20mm}
\pagestyle{empty}
\title{Текст Текст Текст}
\date{\today}
\begin{document}
\maketitle
\tableofcontents
\begin{abstract}
Этот файл -- конспект по органической химии за 1-е полугодие 10-го класса
\end{abstract}
\section{Введение}
УВ - углеводороды\\
Предельный УВ (ПУВ) -- алканы. Вещества, содержащие максимальное
количество $H$, какое только возможно
Гомологи -- вещества, принадлежащие 1-му классу соединений. Имеют схожее
строение и хим. свойства, но отличаются на гомологическую разницу n раз
($CH_{2}$)\\
\paragraph{Правило Зайцева}
$H^{+}$ (водород) отрывается там, где его меньше. А правило Морковникова
говорит, что водород присоединяется туда, где его больше. Положение
галогенов и прочих радикалов в соединении на перемещения водорода
\textbf{никак не влияют}
\subsection{Названия реакция в органической химии}
Гидрирование -- реакиця присоединения водорода; дегидрирование -- обратная
реакция\\
Гидротация -- реакция присоединения воды, дегидротация -- обратная реакция\\
Галогенирование -- реакция присоединения галогена (+Hal с замещением водорода),
дегалогенирование -- обратная реакция\\\\
Соответственно, дегидрогалогенирование -- реакция, в процессе которой от
исходного цещества отсоединяется галоген и водород\\
Полимеризация -- процесс образования высокомолекулярного вещества путем
многократного присоединения мономеров\\
\subsection{Что такое изомеры}
Это вещества, имеющие одинаковый количественный и качественный состав, но
разное строение\\
Изомеризация -- превращение химического соединения в его изомер
\begin{itemize}
\item Структурные
	\begin{itemize}
		\item По углеродной цепи\\
			$CH_{3}-CH_{2}-CH_{2}-CH_{2}-CH_{3}$\\пентан\\\\
			$CH_{3}-CH-CH_{2}-CH_{3}$\\
			2: $-CH_{3}$\\ 2-метилбутан\\\\
			$CH_{3}-{C}-CH_{3}$\\
			2: $-CH_{3}$ x2\\ 2,2-диметилпропан
		\item По положению кратной связи / функциональной группы\\
			$CH_{2}=CH-CH_{2}-CH_{2}-CH_{3}$\\ пентен-1\\\\
			$CH_{3}-CH=CH-CH_{2}-CH_{3}$\\ пентен-2
		\item Межклассовые\\
			$CH_{3}-CH_{2}-CH=CH_{2}$\\ бутен-1\\\\
			$-CH_2-CH_2-CH_2-CH_2-$ (замкнутая формула)\\
			циклобутан
	\end{itemize}
\item Пространственные
	\begin{itemize}
		\item Геометрические
		\item Оптические
	\end{itemize}
\end{itemize}
\pagebreak

\section{Алканы} \begin{flushright} $C_{n}H_{2+2n}$ \end{flushright}
Группа ПУВ, которые содержат только простые одинарные связи
\subsection{Получение}
\subsection{Химические свойства}
	\subsubsection{Горение}
	$C_{4}H_{10}+6.5)_{2}  \xrightarrow{\text{t}} 4CO_{2}+H_{2}O$
	полное окисление
	\subsubsection{Галогенирование}
	$$CH_{4}+Cl-Cl \xrightarrow{h\nu} CH_{3}Cl+HCl \text{(газ)}$$\\
	От метана \textit{откалывается} 1H, от молекулы хлора отпадает 1Cl.
	Получается \textbf{хлорметан} (метилхлорид)
	\subsubsection{Реакция разложения}
	\textit{с метаном}\\
	$CH_{4} \xrightarrow{1000 \degree} C+2H_2$\\
	$2CH_{4} \xrightarrow{1500 \degree} CH \equiv CH+3H_{2}$ ацетилен\\
	\paragraph{Дегидрирование}
	$CH_{3}-CH_{3} \xrightarrow{t \degree,Ni} CH_{2}=CH_{2}+H_{2}$ этилен
	\subsubsection{Изомеризация}
	$CH_{3}-CH_{2}-CH_{2}-CH_{3} \xrightarrow{AlCl_{3}} CH_3-CH-CH_{3}$\\
	2: $-CH_{3}$ (получается 2-метилпропан)
\pagebreak

\section{Алкены} \begin{flushright} $C_{n}H_{2n}$ \end{flushright}
Нерпедельные углеводороды, содержащие 1-у двойную связь.
\subsection{Получение}
	\subsubsection{Дегидрирование алканов}
	$CH_{3}-CH_{2}-CH_{3} \xrightarrow{t \degree, kat} CH_{2}=CH-CH_{3}$
	\subsubsection{Дегидратация спиртов}
	$H-CH_{2}-CH_{2}-OH \xrightarrow{t\degree, H^{+}, -H_{2}O} CH_{2}=CH_{2}$,\\
	где $H^{+}$ -- кислотная среда.
	\subsubsection{Дегидрогалогенирование}
	$CH_{3}-CH-CH_{2}-CH_{3}+NaOH$\\
	2: $-Cl$ 2-хлорбутан\\
	$\xrightarrow{-NaCl,-H2O} CH_{3}-CH=CH-CH_{3}$ бутен-2\\
	От $NaOH$ отделяется $Na$, от хлорбутана -- хлор; ну и вода там
	\subsubsection{Дегалогенирование}
	$CH_{2}-CH-CH_{3}+Zn \xrightarrow{-ZnCl_{2}} CH_{2}=CH-CH{3}$ пропен
\subsection{Химические свойства}
	\subsubsection{Присоединение}
		\paragraph{Гидрирование}
		$CH_{2}=CH_{2}+H_2 \xrightarrow{Ni} CH_3-CH_3$
		\paragraph{Галогенирование}
		$CH_2=CH_2+Cl_2 \xrightarrow{} CH_2-CH_2$
		1,2-дихлорэтан\\
		1, 2: $-Cl$
		\paragraph{Гидрогалогенирование}
		$CH_2=CH-CH_{3}+HBr \xrightarrow{} CH_3-CH-CH_3$ 2-бромпропан\\
		2: $-Br$
		\paragraph{Гидротация}
		$CH_2=CH-CH_3+HOH \xrightarrow{} CH_3-CH-CH_3$ пропанол-2\\
		2: $-OH$
	\subsubsection{Полимеризация}
	$CH_2=CH-CH_3 \xrightarrow{t, p, kat} -CH_2-CH-$\\
	2: $-CH_3$
	\subsubsection{Окисление}
		\paragraph{Полное окисление -- горение}
		$C_{3}H_{6}+0_{2} \xrightarrow{t\degree} 3CO_{2}+3H_{2}O$
		\paragraph{Мягкое окисление}
		$CH_{2}=CH_{2}+[O]+H_{2}O \xrightarrow{} CH_{2}-CH_{2}$
		этан-диол (этиленгликоль)\\
		1,2: $-OH$,\\
		$[O]$ -- окислитель
		\paragraph{Жёсткое окисление}
		$CH_3-CH=CG-CH_3+[O] \xrightarrow{H^{+}} 2CH_{3}COOH$\\
		карбоновая кислота
\pagebreak

\section{Алкодиены} \begin{flushright} $C_{n}H_{2n-2}$ \end{flushright}
(или просто диены) -- непредельные УВ, содержащие в молекуле, кроме
одинарных связей, 2 двйные углерод-углеродные связи\\
Сопряжёнными называются диеновые углеводороды, в молекулах которых 2
двойные $C=C$ связи разделены 1 одинарной связью\\
Вулканизация -- процесс получения резины из каучука
\subsection{Получение}
		$CH_3-CH_2-CH_2-CH_3 \xrightarrow{Pt,p,t} CH_2=CH-CH=CH_2+2H_2$\\
		из н-бутана (нормального бутана) -- бутадиен\\
		$CH_{3}-CH(-CH_3)-CH_{2}-CH_{3} \xrightarrow{Pt,p,t}
		CH_{2}=C(-CH_{3})-CH=CH_{2}+2H_{2}$\\
		из 2-метилбутана -- 2-метилбутадиен-1,3 (изопрен)
\subsection{Химические свойства}
	\subsubsection{Полимеризация}
	$_nCH_{2}=CH-CH=CH_{2} \xrightarrow{kat,t} -(CH_{2}-CH=CH-CH_{2})_n-$\\
	из бутадиена-1,3 -- синтетический бутадиеновый каучук\\\\
	$_nCH_{2}=C(-CH_{3})-CH=CH_2 \xrightarrow{kat,t}
	-(CH_2-C(-CH_3)=CH-CH_2)_n-$\\
	из 2-метилбутадиена-1,3 -- синтетический изопреновый каучук
	\subsubsection{Реакции с присоединением Br}
	$CH_2=CH-CH=CH_2 + Br_2 \xrightarrow{} CH_2(-Br)-CH=CH-CH_2(-Br)$
	из бутадиена-1,3 -- 1,4-дибромбутен-2\\\\
	$CH_2(-Br)-CH=CH-CH_2(-Br) \xrightarrow{}
	CH_2(-Br)-CH(-Br)-CH(-Br)-CH_2(-Br)$\\
	из 1,4-дибромбутена-2 -- 1,2,3,4-тетрабромбутан\\\\
	$CH_2=C-CH=CH_2+Br_2 \xrightarrow{} CH_2(-Br)-C(-CH_3)=CH-CH_2(-Br)$\\
	из 2-метилбутадиена-1,3 -- 1,4-дибром-2-метилбутен-2\\\\
	$CH_2(-Br)-C(-CH_3)=CH-CH_2(-Br)+Br_2 \xrightarrow{}
	CH_2(-Br)-C(-Br)(-CH_3)-CH(-Br)-CH_2(-Br)$\\
	из 1,3-дибром-2-метилбутена-2 -- 1,2,3,4-тетрабром-2-метилбутен\\\\
	$CH_2=CH-CH=CH_2+Br_2 \xrightarrow{} CH_2(-Br)-CH(-Br)-CH=CH_2$\\
	из бутадиена-1,3 -- 1,2-дибромбутен-3.
\pagebreak

\section{Алкины} \begin{flushright} $C_{n}H_{2n-2}$ \end{flushright}
Непредельные УВ, содержащие 1-у тройную связь.
\subsection{Получение}
Гидролиз карбита кальция\\
$CaC_2+2H_2O=Ca(OH)_2+CH\equiv CH$\\
\subsection{Химические свойства}
(Сгорают до $CO_2$ и $H_{2}O$)
	\subsubsection{Присоединение}
		\paragraph{Галогенирование}
		$CH \equiv CH+Br_2 \xrightarrow{} CH(-Br)(-Br)-CH(-Br)(-Br)$\\
		Качественная реакция -- обесцвечивание брома. В
		результате получается 1,1,2,2-тетрабромэтан
		\paragraph{Гидрогалогенирование}
		$CH \equiv C-CH_3+2HBr \xrightarrow{} CH_3-C(-Br)(-Br)-CH_3$
		\paragraph{Гидротация -- реакция Кучерова}
		$CH \equiv CH+H_2O \xrightarrow{Hg,H2SO4_{text{конц.}}}
		CH_3-C(=O)-CH_3$\\
		$CH \equiv C-CH_E+H_2O \xrightarrow{Hg,H_2SO4_{text{конц.}}}
		CH_3-C(=O)-CH_3$\\
		Из этоналя (уксусного ангидрида) получается китон: ацетон
		(пропанон)
		\paragraph{Гидрирование}
		$CH \equiv CH+2H_2 \xrightarrow CH_3-CH_3$
		\paragraph{Изомеризация}
		$3CH\equiv CH \xrightarrow{600\degree C,C_{text{акт.}}}
		C_{6}H_{6} \text{бензол}$\\\\
		$CH \equiv CH+HCl \xrightarrow CH_2=CH(-Cl)$ винил-хлорид\\
\pagebreak

\section{Ароматические УВ} \begin{flushright} $C_{n}H_{2n-6}$ \end{flushright}
К этой группе относятся бензол и его гомологи
\subsection{Получение}
	\subsubsection{Дегидрирование циклоалканов}
	циклогексан $\xrightarrow{t,kat}$ бензол + $3H_2$
	\subsubsection{Полимеризация ацетилена}
	ацетилен $\xrightarrow{C_{\text{акт.}},600\degree}$ 3x -(CH=CH)-
	\subsubsection{Реакция Бюрца}
	$C_6H_5Br + 2Na + BrCH_2-CH_3 = 2NaBr + C_6H_6(-CH_2-CH_3)$
	\subsubsection{Алкелирование}
	$C_6H_6 + CH_3-Cl \xrightarrow{kat,AlCl_3} C_6H_5-CH_3 + HCl$
	метил-бензол (талуол)
\subsection{Химические свойства}
Не окисляется, только сгорает:
$2C_6H_6 + 15O_2 \xrightarrow 12CO_2 + 6H_2O$
	\subsubsection{Реакции замещения}
	\paragraph{Галогенирование}
	$C_6H_6 + Br_2 \xrightarrow{FeBr_3} C_6H_6(-Br)$\\
	$C_6H_6(-CH_3) + Cl_2 \xrightarrow{h\nu} C_6H_6(-CH_2Cl) + HCl$
	\paragraph{Нитрирование}
	$C_6H_6 + HNO_3 \xrightarrow{H_2SO4_{\text{конц.}}} H_2O + C_6H_5(-NO2)$
	\subsubsection{Реакции присоединения}
	$C_6H_6 + 3H_2 \xrightarrow{Ni,\degree t} C_6H_12$\\
	\paragraph{Присоединение галогена}
	$C_6H_6 + 3Cl_2 \xrightarrow{h\nu} C_6H_6Cl_6$ гексохлор-циклогексан\\
	хлорирование

\end{document}
